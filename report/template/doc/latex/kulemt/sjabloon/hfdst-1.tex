\chapter{Het eerste hoofdstuk}
\label{hoofdstuk:1}
Een hoofdstuk behandelt een samenhangend geheel dat min of meer op zichzelf
staat. Het is dan ook logisch dat het begint met een inleiding, namelijk
het gedeelte van de tekst dat je nu aan het lezen bent.

\section{Eerste onderwerp in dit hoofdstuk}
De inleidende informatie van dit onderwerp.

\subsection{Een item}
De bijbehorende tekst. Denk eraan om de paragrafen lang genoeg te maken en
de zinnen niet te lang.

Een paragraaf omvat een gedachtengang en bevat dus steeds een paar zinnen.
Een paragraaf die maar \'e\'en lijn lang is, is dus uit den boze.

\section{Tweede onderwerp in dit hoofdstuk}
Er zijn in een hoofdstuk verschillende onderwerpen. We zullen nu
veronderstellen dat dit het laatste onderwerp is.

\subsection{Een item}
Maak ook geen misbruik van opsommingen. Voor korte opsommingen gebruik je
geen ``\verb|itemize|'' of ``\texttt{enumerate}'' commando's. Doe dus
\emph{niet} het volgende:
\begin{quote}
  De Eiffeltoren bevat drie verdiepingen:
  \begin{itemize}
  \item de eerste;
  \item de tweede;
  \item de derde.
  \end{itemize}
\end{quote}
Maar doe:
\begin{quote}
  De Eiffeltoren bevat drie verdiepingen: de eerste, de tweede en de derde.
\end{quote}

\section{Besluit van dit hoofdstuk}
Als je in dit hoofdstuk tot belangrijke resultaten of besluiten gekomen
bent, dan is het ook logisch om het hoofdstuk af te ronden met een
overzicht ervan. Voor hoofdstukken zoals de inleiding en het
literatuuroverzicht is dit niet strikt nodig.

%%% Local Variables: 
%%% mode: latex
%%% TeX-master: "masterproef"
%%% End: 
